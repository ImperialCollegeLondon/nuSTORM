\section*{Introduction}
This document summarises the steps needed to set-up and run nuSIM.
A summary of the tasks that nuSIM performs may be found
in~\cite{2021:nuSIM:Doc.01}.
nuSIM has been developed in python; python~3 is assumed.

\section*{Getting the code}
nuSIM is maintained using the GitHub version-control system.
The latest release can be downloaded from the
\href{https://www.nustorm.org}{\underline{\color{blue} nuSTORM wiki}}
(\href{https://www.nustorm.org/trac/wiki/Software-and-computing}{\underline{\color{blue}}https://www.nustorm.org/trac/wiki/Software-and-computing}).

\section*{Dependencies and required packages}
nuSIM requires the following packages:
\begin{itemize}
  \item Python modules: \verb+scipy+, \verb+matplotlib+, and
    \verb+pandas+;
  \item CERN programme library: \verb+pyroot+ (which may be installed
    using the standard \verb+root+ installers, see the documentation
    at \href{https://root.cern/install/}{\underline{\color{blue}https://root.cern/install/}}).
\end{itemize}
It may be convenient to run nuSIM in a ``virtual environment''.
To set this up, after updating your python installation to python~3,
and installing root, execute the following commands:
\begin{enumerate}
  \item \verb+python3 -m venv --system-site-packages venv+
    \begin{itemize}
      \item This creates the director \verb+venv+ that contains files
        related to the virtual environment.
    \end{itemize}
  \item \verb+source venv/bin/activate+
  \item \verb+python -m pip install pandas scipy matplotlib+
\end{enumerate}
To exit from the virtual environment, execute the command
\verb+deactivate+. \\
\noindent
The command \verb+source venv/bin/activate+ places you back
into the virtual environment.

\section*{Unpacking the code, directories, and running the tests}
After downloading the package from GitHub, or cloning the repositiry,
you will find a ``\verb+README.md+'' file which provides some orientation
and instructions to run the code.
In particular, a \verb+bash+ script ``\verb+startup.bash+'' is
provided which:
\begin{itemize}
  \item Sets the ``\verb+nuSIMPATH+'' environment variable so that the
    files that hold constants etc. required by the code can be
    located; and
  \item Adds ``\verb+01-Code+'' (see below) to the PYTHONPATH.
    The scripts in "02-Tests" (see below) may then be run with the
    command "python 02-Tests/\textless\,filename\,\textgreater.py".
\end{itemize}
Below the top directory, the directory structure in which the code is
presented is:
\begin{description}
  \item\verb+01-Code+: contains the python implementation as a
    series of modules.
    Each module contains a single class or a related set of methods.
  \item\verb+02-Tests+: contains self-contained test scripts that
    run the various methods and simulation packages defined in the
    code directory.
  \item\verb+11-Parameters+: contains the parameter set used in
    \verb+02-Tests/RunSimulation.py+ to generate muon decays in
    the production straight.
\end{description}
The instruction in the \verb+README.md+ file should be followed to set
up and run the code.

\section{Running the code}
The file in 02-Tests/RunSimulation.py - will run the code and produce a root data  set.\newline

\noindent The file {\bf RunSimulation.py} contains:
\begin{itemize}
\item the definition of the root output file for the generated dataset \newline
rootfilename = os.path.join(nuSIMPATH, 'Scratch/nuSIM-RunSimulation.root')
\item the definition of csv  input file to control the running of the Simulation \newline
filename  = os.path.join(nuSIMPATH, '11-Parameters/nuSTORM-PrdStrght-Params-v1.0.csv')
\item the call to the Simulation class with; the number of events to generate; the central energy to generate; and the filenames\newline
Smltn = Simu.Simulation(5000, 6., filename, rootfilename)
\end{itemize}
\hfill\newline
\noindent Most of the entries for the file {\bf nuSTORM-PrdStrght-Params-v1.0.csv} are self explanatory but it is worth noting:
\begin{itemize}
\item Run Type, rType, 1, i, numSim-2021-01\newline
1 generates a muon decay and 2 generates a pion beam\newline
\item Momentum acceptance,pAcc,10,\%,nuSIM-2021-01\newline
Generates a parabolic distribution with a half width given by the number. For standard generation 10 should be used for pions and 15 muons
\end{itemize}

  
 \subsection{Plots}
 There is a file {\bf 01-Code/Plots.py}. It produces a separate root file with histograms filled by running the programme. The
 plots are written to {\bf plots.root} in the directory from which the job is run. In the default case it will produce a plot
 of the energy of the $\nu_{\mu}$ created by either the $\mu$ or $\pi$ beam. You can either modify this file or produce your own file. The calls are made from {\bf Simulation.py} and Plots.py is included with the line \newline {\it import Plots as plots}
 \hfill\newline
 
 \noindent The class contains three methods: \_\_init\_\_; fill; histdo.\newline
 \_\_init\_\_ is called with no parameters and makes the calls to create a root histogram, the single histogram acts as
 an example\newline
fill(self, $<$ array or class $>$), where the array or class contains the values to be plotted for each event. The example is an array, but it will work  with a class which has suitable {\it get() methods}.\newline
histdo is called with no parameters and writes out all the histograms. The example shows writing out the single histogram
 
 
\section*{Making a contribution}
nuSIM is archived in the git repository \verb+longkr/nuSTORM+.
To clone the code using
\verb+git clone+ you will need your own account on GitHub and
permission to clone the code. 
Instructions to request such permission is posted on the nuSTORM
wiki.
