\newcommand{\dataFormat}{ \noindent \begin{tabular}{|  l | r | c | p{10.0cm} | } \hline Branch: sub-branch & Variable   &  Type  &   Description\\  \hline }
\newcommand{\info}[4]{ {#1}  & {#2}  & {#3} & {#4} \\}
\newcommand{\dataFormatEnd} { \hline \end{tabular}  \nl}
\newcommand{\nl}{\hfill\newline}

\section{Simulation of nuSTORM production straight}
A rudimentary simulation of nuSTORM has been created in python.
The simulation is based on the design presented in~\cite{Ahdida:2020whw}.
To initiate consideration of detector systems and sensitivities the
following parameters have been adopted for the storage ring, the
production straight and the muon-beam optics:
\begin{itemize}
  \item Total ircumference: 616\,m
  \item Length of production straight: 180\,m
  \item Stored muon momentum ($p_\mu$) range: $1 \le p_\mu \le 6$\,GeV
  \item Momentum acceptance: $\pm 15$\%
    \begin{itemize}
      \item Simulate a parabolic momentum distribution with limits
        $\pm 15$\%
    \end{itemize}
  \item Transverse acceptance: 1\,$\pi$\,mm\,rad
  \item Transverse beta function (in both transverse coordinates: 25000\,mm
\end{itemize}
The transverse beta function is taken as “representative” of the
production straight.
It is assumed that $\alpha=0$, and that the emittance, $\epsilon$,
(acceptance) and beta are momentum independent. 
The width of the transverse phase space is obtained using:
\begin{eqnarray}
  x_i        & = & \sqrt{\epsilon \beta} \quad {\rm ; \quad and} \nonumber \\
  x_i^\prime  & = & \sqrt{\frac{\epsilon}{\beta}} \quad ;  \nonumber
\end{eqnarray}
where $x_i$ refers to both transverse coordinates ($x$ and $y$).

\section{Neutrino flux interface specification}
The data is held in root files. 
The TTree is divided into 3 branches: runInfo; beam; and flux.
\begin{description}
  \item {\bf runInfo} is filled once at the beginning of the run and
    contains information about the run.  \\
    At present that is the data format version and the run number. The
    run number should correspond to the number in the name of the
    file.
  \item {\bf beam} is  filled once per event. \\
    The muon beam comes down the nuStorm straight.
    The co-ordinate system has its origin at the beginning of the 
    straight.
    z is the distance down the straight along the machine axis.
    y is the vertical distance from the machine axis and x is the
    horizontal distance completing a right handed co-ordinate system
    (x is to the left as you stand looking down the beam).

    Values of px, py, pz and E are stored for the parent muon and the 
    decay electron, $\nu_{e}$, and $\nu_{\mu}$.

    Decay contains the values relevant to the muon at the decay
    point.
    The x, y, and z co-ordinates in the system defined above are
    stored.
    The angle in the x plane px/pz  (x'), and the angle in the y plane
    (y').
    The distance travelled by the muon measured along its trajectory
    s, at present that is just equal to z.
    The time of the decay.
    This is not stored at present.
    The values of the x, y, x' and y' and $E_{\mu}$ are distributed to
    reflect the energy spread, the $\beta$ and the emittance of the
    beam.
    The model for a particular run can be found by looking in the file
    which contains the description of the simulation conditions for
    each run.
  \item {\bf flux} is  filled once per event. \\
    The neutrinos from the decay are projected forward until they
    intersect with a plane at right angles to the machine axis and
    50\,m beyond the end of the production straight.
    The x, y position at which the neutrino intersects the plane is
    recorded together with the momentum components and the total
    energy of the neutrino.
\end{description}

\section{Data format: version 2.6}
\dataFormat
\info{runInfo:information}{runNumber}{Integer}{Run Number}
\info{}{Version}{Float}{Version number of the data format}
\info{}{FluxPlaneW}{Float}{Unused}
\info{}{FluxPlaneH}{Float}{Unused}
\info{}{PZ}{Float}{Unused}
\info{}{DetectW}{Float}{Unused}
\info{}{DetectH}{Float}{Unused}
\info{}{DetectD}{Float}{Unused}
\info{}{DetectZ}{Float}{Unused}
\info{}{Emit}{Char}{Unused}
\hline
\hline
\info{beam:muon}{px}{Float}{muon momentum in x (horizontal plane)}
\info{}{py}{Float}{muon momentum in y (vertical plane)}
\info{}{pz}{Float}{muon momentum in z (nuStorm machine axis)}
\info{}{E}{Float}{muon Energy}
\hline
\info{beam:decay}{s}{Float}{Decay distance along the particle trajectory. }
\info{}{x}{Float}{x at the decay point, horizontal. Beam centre as origin}
\info{}{y}{Float}{y at the decay point, vertical. Beam centre as origin}
\info{}{z}{Float}{Decay point from the start of the beam straight as origin}
\info{}{xp}{Float}{px/pz at the decay point}
\info{}{yp}{Float}{py/pz at the decay point}
\info{}{t}{Float}{time of the decay - not filled currently}
\dataFormatEnd

\vfill

\dataFormat
\info{beam:electron}{px}{Float}{electron momentum in x (horizontal plane)}
\info{}{py}{Float}{electron momentum in y (vertical plane)}
\info{}{pz}{Float}{electron momentum in z (nuStorm machine axis)}
\info{}{E}{Float}{electron Energy}
\hline
\info{beam:$\nu_{\mu}$}{px}{Float}{$\nu_{\mu}$ momentum in x (horizontal plane)}
\info{}{py}{Float}{$\nu_{\mu}$ momentum in y (vertical plane)}
\info{}{pz}{Float}{$\nu_{\mu}$ momentum in z (nuStorm machine axis)}
\info{}{E}{Float}{$\nu_{\mu}$ Energy}
\hline
\info{beam:$\nu_{e}$}{px}{Float}{$\nu_{e}$ momentum in x (horizontal plane)}
\info{}{py}{Float}{$\nu_{e}$ momentum in y (vertical plane)}
\info{}{pz}{Float}{$\nu_{e}$ momentum in z (nuStorm machine axis)}
\info{}{E}{Float}{$\nu_{e}$ Energy}
\hline
\hline
\info{flux:$\nu_{e}$}{ex}{Float}{x position of intersection of $\nu_{e}$with flux plane}
\info{}{ey}{Float}{y position of intersection of $\nu_{e}$with flux plane}
\info{}{epx}{Float}{px for the $\nu_{e}$ at the flux plane}
\info{}{epy}{Float}{py for the $\nu_{e}$ at the flux plane}
\info{}{epz}{Float}{pz for the $\nu_{e}$ at the flux plane}
\info{}{eE}{Float}{Energy for the $\nu_{e}$ at the flux plane}
\hline
\info{flux:$\nu_{\mu}$}{ex}{Float}{x position of intersection of $\nu_{\mu}$with flux plane}
\info{}{ey}{Float}{y position of intersection of $\nu_{\mu}$with flux plane}
\info{}{epx}{Float}{px for the $\nu_{\mu}$ at the flux plane}
\info{}{epy}{Float}{py for the $\nu_{\mu}$ at the flux plane}
\info{}{epz}{Float}{pz for the $\nu_{\mu}$ at the flux plane}
\info{}{eE}{Float}{Energy for the $\nu_{\mu}$ at the flux plane}
\dataFormatEnd
